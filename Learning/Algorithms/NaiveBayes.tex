\documentclass{article}
\newcommand\tab[1][1cm]{\hspace*{#1}}
\newcommand*\Heq{\ensuremath{\overset{\kern2pt H}{=}}}
\usepackage{amsfonts}
\usepackage[margin=1in]{geometry}
\usepackage[mathscr]{euscript}
\usepackage{enumitem}
\usepackage{amsmath}
\begin{document}
\title{Naive Bayes}
\author{Jason Shipp}
\maketitle

\section*{Concept}
In an ideal world the value of any given feature would be totally independent
from the value of another feature for a given data point. If this were the case
we could use Bayes Law/Rule to compute the missing / predicted value with
certainty after working out the subsequent relationship between the features. As
it stands however this is not the case, and frequently there are complex
underlying relationships in our data that are impossible to model on computers.
It can, however, be usefult to pretend that these relationships are independent
and apply Bayes Law anyway. This is where Naive Bayes comes in! 

\subsection*{Multinomial Naive Bayes}


\section*{Math Involved}

\subsection*{Bayes Rule} \\
\begin{equation}
  \text{Let } x = (x_1, x_2, ..., x_n) \\
  p(C_k | x) = \frac{p(C_k)p(x | C_k)}{p(x)}\\
  \text{Read } p(y|x) \text{as the probility of y given that x has occurred}
\end{equation}

\subsection*{Naive Bayes Assumption}
\begin{equation}
  C_k = argmax p(C_k)\prod_{i=1}^{n}p(x_k | C_k)
\end{equation}

\section*{Code}
\subsection*{Scikit-Learn}
% TODO block off correct area
\lstinputlisting[language=Python]{../Code/NaiveBayes.py} 
